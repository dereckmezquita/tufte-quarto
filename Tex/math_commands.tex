\usepackage{bm}
\usepackage{bigints}
\usepackage{nicefrac}
\usepackage{venndiagram}
\usepackage{xpatch}


%  SumInt operator
\DeclareMathOperator*{\SumInt}{%
\mathchoice%
  {\ooalign{$\displaystyle\sum$\cr\hidewidth$\displaystyle\bigintsss$\hidewidth\cr}}
  {\ooalign{\raisebox{.14\height}{\scalebox{.7}{$\textstyle\sum$}}\cr\hidewidth$\textstyle\bigintsss$\hidewidth\cr}}
  {\ooalign{\raisebox{.2\height}{\scalebox{.6}{$\scriptstyle\sum$}}\cr$\scriptstyle\bigintsss$\cr}}
  {\ooalign{\raisebox{.2\height}{\scalebox{.6}{$\scriptstyle\sum$}}\cr$\scriptstyle\bigintsss$\cr}}
}


% https://tex.stackexchange.com/questions/218372/cancel-maths-expression-with-a-down-arrow
\makeatletter
% #1, #2 offset of label   #6 extra width to clear arrowhead
% #3, #4 vector direction  #7 superscript label style
% #5 vector width          #8 superscript label
\def\cantox@vector#1#2#3#4#5#6#7#8{%
  \dimen@.5\p@
  \setbox\z@\vbox{\boxmaxdepth.5\p@
   \hbox{\kern-1.2\p@\kern#1\dimen@$#7{#8}\m@th$}}%
  \ifx\canto@fil\hidewidth  \wd\z@\z@ \else \kern-#6\unitlength \fi
  \ooalign{%
    \canto@fil$\m@th \CancelColor
    \vcenter{\hbox{\dimen@#6\unitlength \kern\dimen@
      \multiply\dimen@#4\divide\dimen@#3 \vrule\@depth\dimen@\@width\z@
      \vector(#3,-#4){#5}%
    }}_{\raise-#2\dimen@\copy\z@\kern-\scriptspace}$%
    \canto@fil \cr
    \hfil \box\@tempboxa \kern\wd\z@ \hfil \cr}}
\def\bcancelto#1#2{\let\canto@vector\cantox@vector\cancelto{#1}{#2}}
\makeatother

% Remove venndiagram frame
\tikzset{
  vennframe/.style={draw=none} % define a new style for the frames
}
\makeatletter
\xpatchcmd{\endvenndiagram3sets}
{\draw (0,0) rectangle (\@venn@w,\@venn@h);} % replace this
{\draw [vennframe] (0,0) rectangle (\@venn@w,\@venn@h);} % with this -- add the style
{}{}

\xpatchcmd{\endvenndiagram2sets}
{\draw (venn bottom left) rectangle (\@venn@w,\@venn@h);}
{\draw [vennframe] (venn bottom left) rectangle (\@venn@w,\@venn@h);}
{}{}
\makeatother

\newcommand*\wrapscope[1]{%
  \expandafter\newcommand\csname o#1\endcsname[1][]{%
    \begin{scope}[##1]
    \csname #1\endcsname
    \end{scope}
  }%
}%

\wrapscope{fillA}
\wrapscope{fillACapB}

\makeatletter
\tikzset{
  clip/.append code={%
    \let\tikz@options=\pgfutil@empty
    \tikz@addmode\tikz@mode@fillfalse%
    \tikz@addmode\tikz@mode@drawfalse%
    \tikz@addmode\tikz@mode@doublefalse%
    \tikz@addmode\tikz@mode@boundaryfalse%
    \tikz@addmode\tikz@mode@fade@pathfalse%
    \tikz@addmode\tikz@mode@fade@scopefalse%
  }
}
\makeatother




\DeclareSymbolFont{bbold}{U}{bbold}{m}{n}
\DeclareSymbolFontAlphabet{\mathbbold}{bbold}



\newcommand{\diag}{\mathop{\mathrm{diag}}}
\newcommand{\typical}[1]{{\sA^{\epsilon}_{#1} }}
\newcommand{\Typical}[1]{{\sA^{\delta}_{#1} }}

\newcommand{\ind}{\bot}
% Highlight a newly defined term
\newcommand{\newterm}[1]{{\bf #1}}
% \newcommand{\eqdef}{{\overset{\text{def}}{=}}}
\newcommand{\eqdef}{{\triangleq}}

% \let\oldint\int
% \renewcommand{\int}{{\bigintssss}}
\renewcommand{\leq}{{~\leqslant ~}}
\renewcommand{\geq}{{~\geqslant ~}}


\newcommand{\train}{\mathcal{D}}
\newcommand{\valid}{\mathcal{D_{\mathrm{valid}}}}
\newcommand{\test}{\mathcal{D_{\mathrm{test}}}}