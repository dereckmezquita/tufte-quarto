
% DEFINITIONS


% The fancyvrb package lets us customize the formatting of verbatim
% environments.  We use a slightly smaller font.
\usepackage{fancyvrb}
\fvset{fontsize=\normalsize}

%%
% Prints argument within hanging parentheses (i.e., parentheses that take
% up no horizontal space).  Useful in tabular environments.
\newcommand{\hangp}[1]{\makebox[0pt][r]{(}#1\makebox[0pt][l]{)}}

%%
% Prints an asterisk that takes up no horizontal space.
% Useful in tabular environments.
\newcommand{\hangstar}{\makebox[0pt][l]{*}}

%%
% Prints a trailing space in a smart way.
\usepackage{xspace}

%%
% Some shortcuts for Tufte's book titles.  The lowercase commands will
% produce the initials of the book title in italics.  The all-caps commands
% will print out the full title of the book in italics.
\newcommand{\vdqi}{\textit{VDQI}\xspace}
\newcommand{\ei}{\textit{EI}\xspace}
\newcommand{\ve}{\textit{VE}\xspace}
\newcommand{\be}{\textit{BE}\xspace}
\newcommand{\VDQI}{\textit{The Visual Display of Quantitative Information}\xspace}
\newcommand{\EI}{\textit{Envisioning Information}\xspace}
\newcommand{\VE}{\textit{Visual Explanations}\xspace}
\newcommand{\BE}{\textit{Beautiful Evidence}\xspace}

\newcommand{\TL}{Tufte-\LaTeX\xspace}

% Prints the month name (e.g., January) and the year (e.g., 2008)
\newcommand{\monthyear}{%
  \ifcase\month\or January\or February\or March\or April\or May\or June\or
  July\or August\or September\or October\or November\or
  December\fi\space\number\year
}


% Prints an epigraph and speaker in sans serif, all-caps type.
\newcommand{\epigraph}[2]{%
  \begin{fullwidth}
  \begin{flushright}
  \sffamily\fontsize{8}{10}\selectfont
  \sffamily\footnotesize
  \begin{doublespace}
  \vspace{-8cm}\noindent\allcaps{#1}\\% epigraph
  \noindent\allcaps{#2}\\% author
  \end{doublespace}
  \vspace{5.1cm}
  \end{flushright}
  \end{fullwidth}
  \normalfont
}


\newcommand{\blankpage}{\newpage\hbox{}\thispagestyle{empty}\newpage}


% insert 4cm before quote
\renewenvironment{quote}{
  \list{}{\leftmargin=3.5cm\topsep=0pt}
  \item\relax\small\itshape
}
{\endlist}


%  change chapter formatting
\titlespacing*{\chapter}{0pt}{5cm}{1cm}
% \titlespacing*{\section}{0pt}{.6em}{.3em}
% \titlespacing*{\subsection}{0pt}{.4em}{.2em}

\titlespacing*{\section}{0pt}{0cm}{0cm}
\titlespacing*{\subsection}{0pt}{0cm}{0cm}


%  Change Figure Caption in the Margin size
% \renewenvironment{@tufte@margin@float}[2][-1.2ex]%
%   {\FloatBarrier% process all floats before this point so the figure/table numbers stay in order.
%   \begin{lrbox}{\@tufte@margin@floatbox}%
%   \begin{minipage}{\marginparwidth}%
%     \@tufte@caption@font\footnotesize% <-- Add fontnotesize
%     \def\@captype{#2}%
%     \hbox{}\vspace*{#1}%
%     \@tufte@caption@justification%
%     \@tufte@margin@par%
%     \noindent\normalsize%<-- restored size
%   }
%   {\end{minipage}%
%   \end{lrbox}%
%   \marginpar{\usebox{\@tufte@margin@floatbox}}%
  % }


\renewcommand\footnotesize{%
   \@setfontsize\footnotesize\@viiipt{9}%
   \abovedisplayskip 5\p@ \@plus2\p@ \@minus4\p@
   \abovedisplayshortskip \z@ \@plus\p@
   \belowdisplayshortskip 2.8\p@ \@plus\p@ \@minus2\p@
   \def\@listi{\leftmargin\leftmargini
               \topsep 2.5\p@ \@plus\p@ \@minus\p@
               \parsep 2\p@ \@plus\p@ \@minus\p@
               \itemsep \parsep}%
   \belowdisplayskip \abovedisplayskip
}

% % Define Tuftian float styles (with the caption in the margin)
% \newcommand{\floatc@tufteplain}[2]{%
% \begin{lrbox}{\@tufte@caption@box}%
%   \begin{minipage}[\floatalignment]{\marginparwidth}\hbox{}%
%     \footnotesize\@tufte@caption@font{\@fs@cfont #1:} #2\par\normalsize%
%   \end{minipage}%
% \end{lrbox}%
% \smash{\hspace{\@tufte@caption@fill}\usebox{\@tufte@caption@box}}%
% }